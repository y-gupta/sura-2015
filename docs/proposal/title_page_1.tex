\documentclass[12pt]{article}
\usepackage[pdftex]{graphicx}
\usepackage{url}

% These are additional packages for "pdflatex", graphics, and to include
% hyperlinks inside a document.

\setlength{\oddsidemargin}{0.25in}
\setlength{\textwidth}{6.5in}
\setlength{\topmargin}{0in}
\setlength{\textheight}{8.5in}

\title{SURA-Proposal}
\author{Yash Gupta}

\begin{document}

\begin{titlepage}

\newcommand{\HRule}{\rule{\linewidth}{0.5mm}} % Defines a new command for the horizontal lines, change thickness here

\center % Center everything on the page

\textsc{\LARGE Indian Institute of Technology, Delhi}\\[1.0cm] 
\textsc{\Large Summer Research Project Proposal - 2015}\\[1.0cm] 
\includegraphics[height=4cm]{iitd_logo.png}\\[2.0cm] 
\HRule \\[0.5cm]
{ \huge \bfseries Metal Corrosion}\\[0.4cm] % Title of your document
\textsc{\large Collection, Simulation and Rendering}\\[0.2cm] % Minor heading such as course title
\HRule \\[3.5cm]
 

\begin{minipage}{0.45\textwidth}
\begin{flushleft} \large
\emph{Author:}\\
\textbf{Yash Gupta}\\
\emph{Department:} CSE\\
\emph{Entry no.:} 2013CS10302\\
\emph{Contact:} +91-85888-24291\\
\emph{CGPA:} 9.31
\end{flushleft}
\end{minipage}
~
\begin{minipage}{0.45\textwidth}
\begin{flushright} \large
\emph{Facilitator:} \\
\textbf{Prof. Subodh Kumar}\\
Associate Professor, CSE\\[0.5cm]
\emph{Head of Deptartment:}\\
\textbf{Prof. Saroj Kaushik}\\
Head, CSE
\end{flushright}
\end{minipage}\\[1cm]

\vfill % Fill the rest of the page with whitespace

\end{titlepage}

\section{Objectives}

The complete pipeline of interest to the project is:

\begin{itemize}
\item Collecting images of corroded metal blocks, along with material properties (alloy composition, surface treatments, etc)
\item Filtering out external effects from the images (lightning, noise, etc.)
\item Mapping scalar corrosion and weathering levels on the surface of the metal block.
\item Physical modeling of the corrosion process and simulating it over a time series.
\item Realistic rendering of the corroded and weathered object from various maps obtained from previous steps.
\end{itemize}

This project aims particularly at realistic rendering of the corroded object.

\section{Approach}

\subsection{Data Collection}

Collection of visual properties of surface will be done with regular photographs of corroded blocks.\\
The photographs will be filtered to eliminate specular lightning effects and obtain a diffuse color map of the surface.\\
The images will be mapped to the surfaces in reconstructed 3d version.\\
A user will mark areas in the image with a degree of corrosion and complete scalar corrosion map will be calculated from it.\\

Highly detailed photographs and depth maps are available due to ongoing research in the same at IITD itself, and can be used as starting points.

\subsection{Physical Modeling}

The metallic object can be modeled as an array of voxels (small volume elements), each voxel acting as a cell with properties like corrosion level, diffuse color, albedo (reflectivity).\\
Corrosion being a localized surface phenomena, will be modeled using cellular automaton.\\
Effects of factors like weathering, alloy composition and humidity can be incorporated into the automaton parameters relating a cell's properties with neighboring cells.\\

Research has been done on uniform and pit corrosion modeling at IIT Delhi itself, and can be used to get physically accurate results.

\subsection{Simulation}

The model generated in the previous step will be simulated over finite time steps.\\
Various maps (diffuse color map, normal map, albedo map) are generated for the surface in this stage, using the final properties of voxels after simulation.

\subsection{Rendering}

The object mesh will be generated from the voxel-array representation.\\
Diffuse map, normal/bump map and albedo map generated in the previous steps will be used with specifically programmed pixel and vertex shaders to render a realistic representation of the corroded and weathered object.

This area has not been extensively explored in ongoing research at IITD, and thus potential improvements can be made. The present method is to use proprietary software (3DS-Max) to render the final results, however it is not configurable to the specific needs of corrosion and weathering effects.

% \subsection{Existing Methods}
% Existing methods rely on 

\section{Duration}
The scope of research is vast in almost all components (Collection, Modeling and Rendering) of the project, however, significant results can be expected over the summer vacation (8 weeks). 


\section{Facilities}

The project requires:
\begin{itemize}
\item Camera(s) for collection of surface images. (Existing data from ongoing research can also be used)
\item Computation facilities for simulation and rendering part.
\end{itemize}
Otherwise, no special facilities are required by the project.


\section{Budget}

The project does not inherently require a budget, although better collection of surface details can be done with laser-based depth scanning (Existing data can also be used here).\\
Existing computing facilities in the institute can be used for Simulation and Rendering.


\section{References}

\url{http://www.dtic.mil/cgi-bin/GetTRDoc?AD=ADA508789}

\end{document}